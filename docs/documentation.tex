\documentclass{article}
\usepackage[dvipsnames]{xcolor}
\usepackage[utf8]{inputenc}
\usepackage{appendix}
\usepackage[T1]{fontenc}
\usepackage[italian]{babel}
\usepackage{siunitx} % Provides the \SI{}{} and \si{} command for typesetting SI units
\usepackage{graphicx} % Required for the inclusion of images
\usepackage{natbib} % Required to change bibliography style to APA
\usepackage{amsmath} % Required for some math elements
\usepackage{caption}
\usepackage{tikz}
\usepackage{float}
\usetikzlibrary{arrows,automata, positioning}
\usepackage{import}
\setlength\parindent{0pt} % Removes all indentation from paragraphs

\title{\vspace{-2cm}Progetto di Tecnologie Informatiche per il Web} % Title
\author{Negri Riccardo} % Author name
\date{\today}

\begin{document}
	
	\maketitle
	
	\begin{figure}[H]
		\centering
		\includegraphics[width=0.365\textwidth]{assets/logo.jpg}
	\end{figure}

	\begin{center}
		\begin{tabular}{l l l l}
			Docente: & Fraternali Piero& & \\ 
			Studente: & Negri Riccardo & 10729927 & 936820 
		\end{tabular}
	\end{center}
	
	\tableofcontents
	\pagebreak
	
	\section{Specifiche}
	
	\subsection{Versione pure HTML}
		Un’applicazione web consente la gestione di trasferimenti di denaro online da un conto a un
		altro. L’applicazione supporta registrazione e login mediante una pagina pubblica con
		opportune form. La registrazione controlla la validità sintattica dell’indirizzo di email e
		l’uguaglianza tra i campi “password” e “ripeti password”. La registrazione controlla l’unicità
		dello username. Un utente ha un nome, un cognome, uno username e uno o più conti correnti.
		Un conto ha un codice, un saldo, e i trasferimenti fatti (in uscita) e ricevuti (in ingresso) dal
		conto. Un trasferimento ha una data, un importo, un conto di origine e un conto di destinazione.
		Quando l’utente accede all’applicazione appare una pagina LOGIN per la verifica delle
		credenziali. In seguito all’autenticazione dell’utente appare l’HOME page che mostra l’elenco
		dei suoi conti. Quando l’utente seleziona un conto, appare una pagina STATO DEL CONTO
		che mostra i dettagli del conto e la lista dei movimenti in entrata e in uscita, ordinati per data
		discendente. La pagina contiene anche una form per ordinare un trasferimento. La form
		contiene i campi: codice utente destinatario, codice conto destinatario, causale e importo.
		All’invio della form con il bottone INVIA, l’applicazione controlla che il conto di destinazione
		appartenga all’utente specificato e che il conto origine abbia un saldo superiore o uguale
		all’importo del trasferimento. In caso di mancanza di anche solo una condizione, l’applicazione
		mostra una pagina con un avviso di fallimento che spiega il motivo del mancato trasferimento.
		Nel caso in cui entrambe le condizioni siano soddisfatte, l’applicazione deduce l’importo dal
		conto di origine, aggiunge l’importo al conto di destinazione e mostra una pagina CONFERMA
		TRASFERIMENTO che presenta i dati dell’importo trasferito e i dati del conto di origine e di
		destinazione con i rispettivi saldi precedenti al trasferimento e aggiornati dopo il trasferimento.
		L’applicazione deve garantire l’atomicità del trasferimento: ogni volta che il conto di
		destinazione viene addebitato, il conto di origine deve essere accreditato. Ogni pagina
		contiene un collegamento per tornare alla pagina precedente. L’applicazione consente il
		logout dell’utente.
	
	\pagebreak
	\subsection{Versione RIA}
		Si realizzi un’applicazione client server web che modifica le specifiche precedenti come segue:
		\begin{itemize}
		\item La registrazione controlla la validità sintattica dell’indirizzo di email e l’uguaglianza tra
		i campi “password” e “ripeti password”, anche a lato client.
		\item Dopo il login, l’intera applicazione è realizzata con un’unica pagina.
		\item 	Ogni interazione dell’utente è gestita senza ricaricare completamente la pagina, ma
		produce l’invocazione asincrona del server e l’eventuale modifica del contenuto da
		aggiornare a seguito dell’evento.
		\item I controlli di validità dei dati di input (ad esempio importo non nullo e maggiore di zero)
		devono essere realizzati anche a lato client.
		\item L’avviso di fallimento è realizzato mediante un messaggio nella pagina che ospita
		l’applicazione.
		\item L’applicazione chiede all’utente se vuole inserire nella propria rubrica i dati del
		destinatario di un trasferimento andato a buon fine non ancora presente. Se l’utente
		conferma, i dati sono memorizzati nella base di dati e usati per semplificare
		l’inserimento. Quando l’utente crea un trasferimento, l’applicazione propone mediante
		una funzione di auto-completamento i destinatari in rubrica il cui codice corrisponde
		alle lettere inserite nel campo codice utente destinatario.
		\end{itemize}
	
	\pagebreak
	\section{Database design}
	Legenda: \textcolor{red}{entità}, \textcolor{ForestGreen}{attributi}, \textcolor{blue}{relazioni}.
	\\
	\\
	Un’applicazione web consente la gestione di trasferimenti di denaro online da un conto a un
	altro. L’applicazione supporta registrazione e login mediante una pagina pubblica con
	opportune form. La registrazione controlla la validità sintattica dell’indirizzo di email e
	l’uguaglianza tra i campi “password” e “ripeti password”. La registrazione controlla l’unicità
	dello username. Un utente ha un nome, un cognome, uno username e uno o più conti correnti.
	Un conto ha un codice, un saldo, e i trasferimenti fatti (in uscita) e ricevuti (in ingresso) dal
	conto. Un trasferimento ha una data, un importo, un conto di origine e un conto di destinazione.
	Quando l’utente accede all’applicazione appare una pagina LOGIN per la verifica delle
	credenziali. In seguito all’autenticazione dell’utente appare l’HOME page che mostra l’elenco
	dei suoi conti. Quando l’utente seleziona un conto, appare una pagina STATO DEL CONTO
	che mostra i dettagli del conto e la lista dei movimenti in entrata e in uscita, ordinati per data
	discendente. La pagina contiene anche una form per ordinare un trasferimento. La form
	contiene i campi: codice utente destinatario, codice conto destinatario, causale e importo.
	All’invio della form con il bottone INVIA, l’applicazione controlla che il conto di destinazione
	appartenga all’utente specificato e che il conto origine abbia un saldo superiore o uguale
	all’importo del trasferimento. In caso di mancanza di anche solo una condizione, l’applicazione
	mostra una pagina con un avviso di fallimento che spiega il motivo del mancato trasferimento.
	Nel caso in cui entrambe le condizioni siano soddisfatte, l’applicazione deduce l’importo dal
	conto di origine, aggiunge l’importo al conto di destinazione e mostra una pagina CONFERMA
	TRASFERIMENTO che presenta i dati dell’importo trasferito e i dati del conto di origine e di
	destinazione con i rispettivi saldi precedenti al trasferimento e aggiornati dopo il trasferimento.
	L’applicazione deve garantire l’atomicità del trasferimento: ogni volta che il conto di
	destinazione viene addebitato, il conto di origine deve essere accreditato. Ogni pagina
	contiene un collegamento per tornare alla pagina precedente. L’applicazione consente il
	logout dell’utente.
	
	\section{Versione pure HTML}
	
	\section{Versione RIA}
	
	
\end{document}